\documentclass[12pt]{jsarticle}
\usepackage[dvipdfmx]{graphicx}
\textheight = 25truecm
\textwidth = 18truecm
\topmargin = -1.5truecm
\oddsidemargin = -1truecm
\evensidemargin = -1truecm
\marginparwidth = -1truecm

\def\theenumii{\Alph{enumii}}
\def\theenumiii{\alph{enumiii}}
\def\labelenumi{(\theenumi)}
\def\labelenumiii{(\theenumiii)}
\def\theenumiv{\roman{enumiv}}
\def\labelenumiv{(\theenumiv)}
\usepackage{comment}
\usepackage{url}

%%%%%%%%%%%%%%%%%%%%%%%%%%%%%%%%%%%%%%%%%%%%%%%%%%%%%%%%%%%%%%%%
%% sty/ にある研究室独自のスタイルファイル
\usepackage{jtygm}  % フォントに関する余計な警告を消す
\usepackage{nutils} % insertfigure, figref, tabref マクロ

\def\figdir{./figs} % 図のディレクトリ
\def\figext{pdf}    % 図のファイルの拡張子

\begin{document}
%%%%%%%%%%%%%%%%%%%%%%%%%%%%
%% 表題
%%%%%%%%%%%%%%%%%%%%%%%%%%%%
\begin{center}
{\LARGE SlackBotプログラム作成の報告書}
\end{center}

\begin{flushright}
  2020/6/24\\
  三宅 貴義
\end{flushright}
%%%%%%%%%%%%%%%%%%%%%%%%%%%%
%% 概要
%%%%%%%%%%%%%%%%%%%%%%%%%%%%
\section{はじめに}
\label{sec:introduction}
本資料は,2020年度B4新人研修課題であるSlackBotプログラム作成の報告書である.


\section{作成した機能}\label{sec:hoge}
作成したSlackBotプログラムは以下の3つの機能を持つ.

\begin{enumerate}
  \item 文字列を返信する機能\\
        ユーザが“(文字列)と言って”と投稿した際に,“(文字列)”と返信する機能.
  \item 郵便番号を住所に変換する機能\\
        ユーザが“(郵便番号)の住所”と投稿した際に,(郵便番号)を対応する住所に変換して返信する機能
  \item コンビニを検索する機能\\
        ユーザが“(住所)のコンビニを(件数)件検索”と投稿した際に,
        (住所)の近くにあるコンビニの名前とGoogle MapのURLを(件数)分返信する機能
\end{enumerate}


\section{理解できなかった部分}
本課題を進めていく中で,理解できなかった部分を以下に示す.

\begin{enumerate}
  \item sinatraの仕組み\\
        SlackBotプログラムはWebアプリケーションフレームワークであるsinatraを用いて動作している.
        これがどうのような仕組みなのかが理解できなかった.
\end{enumerate}

\section{作成できなかった機能}
作成できなかった機能を以下に示す.

\begin{enumerate}
  \item 検索したコンビニの画像や,画像のURLを返信する機能
\end{enumerate}

\section{自主的に作成した機能}
課題として提示されていないが,自主的に作成した機能を以下に示す.

\begin{enumerate}
  \item プログラム作成者以外のSlackアカウントからSlackBotサーバへPOSTされた際に,返信をしない機能
  \item 投稿をしたユーザへのメンションをつけて,返信する機能
\end{enumerate}


\section{おわりに}
\label{sec:conclusion}
本資料では,SlackBotプログラム作成の報告を行った.

%\bibliographystyle{ipsjunsrt}
%\bibliography{mybibdata}

\end{document}
