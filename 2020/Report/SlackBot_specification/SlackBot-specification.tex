\documentclass[12pt]{jsarticle}
\usepackage[dvipdfmx]{graphicx}
\textheight = 25truecm
\textwidth = 18truecm
\topmargin = -1.5truecm
\oddsidemargin = -1truecm
\evensidemargin = -1truecm
\marginparwidth = -1truecm

\def\theenumii{\Alph{enumii}}
\def\theenumiii{\alph{enumiii}}
\def\labelenumi{(\theenumi)}
\def\labelenumiii{(\theenumiii)}
\def\theenumiv{\roman{enumiv}}
\def\labelenumiv{(\theenumiv)}
\usepackage{comment}
\usepackage{url}

%%%%%%%%%%%%%%%%%%%%%%%%%%%%%%%%%%%%%%%%%%%%%%%%%%%%%%%%%%%%%%%%
%% sty/ にある研究室独自のスタイルファイル
\usepackage{jtygm}  % フォントに関する余計な警告を消す
\usepackage{nutils} % insertfigure, figref, tabref マクロ

\def\figdir{./figs} % 図のディレクトリ
\def\figext{pdf}    % 図のファイルの拡張子

\begin{document}
%%%%%%%%%%%%%%%%%%%%%%%%%%%%
%% 表題
%%%%%%%%%%%%%%%%%%%%%%%%%%%%
\begin{center}
{\LARGE SlackBotプログラムの仕様書(修正版)}
\end{center}

\begin{flushright}
  2020/6/24\\
  三宅 貴義
\end{flushright}
%%%%%%%%%%%%%%%%%%%%%%%%%%%%
%% 概要
%%%%%%%%%%%%%%%%%%%%%%%%%%%%
\section{修正箇所}
\begin{enumerate}
  \item “任意の”という言葉を“指定の”に修正した.
  \item \ref{sec:flor}章で,処理の流れを表す図を追加した.
  \item \ref{sec:function}章で,機能の名前を統一した.
  \item \ref{sec:flor}章で,実際のやり取りを図に変更した.
  \item \ref{sec:envi}章で,設定を説明する順番を修正した.
  \item \ref{sec:envi}章で,Incoming Webhook及び,Outgoing Webhookをアプリケーションと言及していたことを修正.
  \item “Github”,“Incoming Webhooks”や“Place API”などの誤記を修正した.
  \item その他の日本語の間違いを修正した.
\end{enumerate}

\section{はじめに}
\label{sec:introduction}
本資料は2020年度B4新人研修課題のSlackBotプログラムの仕様についてまとめたものである.
SlackBotとは,Slack上に特定のメッセージが投稿された際に,内容に応じた返信を自動で行うプログラムである.
本プログラムは以下の3つの機能を持つ.
\begin{enumerate}
  \item 文字列を返信する機能
  \item 郵便番号を住所に変換する機能
  \item コンビニを検索する機能
\end{enumerate}

%%%%%%%%%%%%%%%%%%%%%%%%%%%%%%%%%%%%%%%
%%%%%%%%%%%%%%%%%%%%%%%%%%%%%%%%%%%%%%%
\section{対象となる利用者}
本プログラムを利用するには,以下の3つのアカウントを所有している必要がある.

\begin{enumerate}
  \item Slackアカウント
  \item Googleアカウント
  \item クレジットカードアカウント
\end{enumerate}

なお,クレジットカードはGoogle Places APIを利用するために必要であり,
APIのリクエストが規定値を超えると請求が発生する.


%%%%%%%%%%%%%%%%%%%%%%%%%%%%%%%%%%%%%%%
%%%%%%%%%%%%%%%%%%%%%%%%%%%%%%%%%%%%%%% 
\section{SlackBotプログラムの処理の流れ}\label{sec:flor}
SlackBotプログラムの処理の流れを\figref{access-map}に示す.

\insertfigure[0.9]{access-map}{access-map}{SlackBotプログラムの処理の流れ}{Stream of SlackBot program.}

また,\figref{access-map}の処理について以下に説明を示す.
\begin{enumerate}
  \item ユーザがSlackサーバにメッセージを投稿する.
  \item SlackBotサーバは,Outgoing Webhookの機能で,Slackサーバから投稿内容を受け取る.
  \item SlackBotサーバは,投稿内容を処理し,リクエストを作成する.
  \item SlackBotサーバは,APIにリクエストを送信する.
  \item SlackBotサーバは,APIから返却されたレスポンスを受け取る.
  \item SlackBotサーバは,レスポンスを処理し,メッセージを作成する.
  \item SlackBotサーバは,SlackサーバへIncoming Webhookを利用してメッセージを送信する.
  \item ユーザがSlackBotサーバのメッセージを返信として受け取る.
\end{enumerate}

%%%%%%%%%%%%%%%%%%%%%%%%%%%%%%%%%%%%%%%%%
%%%%%%%%%%%%%%%%%%%%%%%%%%%%%%%%%%%%%%%%%
\section{機能}
\label{sec:function}
この章ではSlackBotプログラムの機能について説明を行う.

\subsection{文字列を返信する機能}
この機能はユーザーが指定した文字列を返信する機能である.
この機能を使用するにはSlackに以下のように投稿する.\\
@MiyakeBot 「$\langle$文字列$\rangle$」と言って\\
$\langle$文字列$\rangle$には返信させる文字列を入力する.\\
具体的なやり取りを\figref{hi}に示す.

  \insertfigure[0.7]{hi}{hi}{文字列を返信する機能のやり取り}{Stream of SlackBot program.}


\subsection{郵便番号を住所に変換する機能}
この機能はユーザが指定した郵便番号を,住所に変換して返信する機能である.
この機能を使用するにはSlackに以下のように投稿する.\\
@MiyakeBot 「$\langle$郵便番号$\rangle$」の住所を検索\\
$\langle$郵便番号$\rangle$には検索する郵便番号を入力する.\\
具体的なやり取りを\figref{address}に示す.

  \insertfigure[0.7]{address}{address}{郵便番号を検索する機能のやり取り}{Stream of SlackBot program.}


\subsection{コンビニを検索する機能}
この機能はユーザが指定した場所に関する情報から,その近辺のコンビニを指定した件数だけ表示する機能である.
この機能を使用するには以下のように投稿する.\\
@MiyakeBot 「$\langle$場所$\rangle$」のコンビニを$\langle$件数$\rangle$件検索\\
$\langle$場所$\rangle$には検索する場所を,$\langle$件数$\rangle$には表示する件数を数字で入力する.\\
具体的なやり取りを\figref{conveni}に示す.

  \insertfigure[0.7]{conveni}{conveni}{コンビニを検索する機能のやり取り}{Stream of SlackBot program.}


%%%%%%%%%%%%%%%%%%%%%%%%%%%%%%%%%%%%%%%%%%%
%%%%%%%%%%%%%%%%%%%%%%%%%%%%%%%%%%%%%%%%%%%
\section{動作環境}\label{sec:envi}
本プログラムはHerokuで動作する.
Herokuとはクラウド上でアプリケーションの実行が可能なPaaSである.
Herokuの環境を表\ref{tab:env_1}に示す.

\begin{table}[tb]
  \begin{center}
    \caption{動作環境}\label{tab:env_1}
    \ecaption{Working environment.}
    \begin{tabular}{l|l}
      \hline\hline
      項目 & 内容 \\\hline
      CPU & Intel(R) Xeon(R) CPU E5-2670 v2 @2.50GHz\\
      メモリ & 512MB\\
      OS & Ubuntu 18.04.4 LST\\
      Rub & ruby 2.6.6p146\\
      Gem & activesupport 6.0.2.2\\
       & bundler 2.0.2\\
       & rack 2.2.2\\
       & rack-protection 2.0.8.1 \\
       & ruby2\_keyeords 0.0.2\\
       & sinatra 2.0.8.1\\
       & tilt 2.0.10\\
       & mustermann 1.1.1\\
      \hline
    \end{tabular}
  \end{center}
\end{table}


%%%%%%%%%%%%%%%%%%%%%%%%%%%%%%%%%%%%%%%%%%
%%%%%%%%%%%%%%%%%%%%%%%%%%%%%%%%%%%%%%%%%%
\section{環境構築}
本プログラムを利用するには以下の手順に従って設定を行う.

\begin{enumerate}
  \item 本プログラムの入手
  \item Incoming Webhookの設定
  \item Google Places APIの設定
  \item Herokuの設定 
  \item Outgoing Webhookの設定
\end{enumerate}

それぞれの項目について,詳細を説明する.

%%%%%%%%%%%%%%%%%%%%%%%%%%%%%%%%%%%%%%%%%%%%%%%%%%%%%%%%%%%%%%%%%%%%%%%%
\subsection{本プログラムの入手}
本プログラムはGitHub上で公開,管理されている.
本プログラムを利用するには以下のコマンドを実行し,BootCampレポジトリをクローンする.\\
\verb|$ git clone https://github.com/miyake13000/BootCamp|

クローンしたレポジトリ内のSlackBotディレクトリを作業ディレクトリに変更する.\\
\verb|$ cd BootCamp/2020/SlackBot|

%%%%%%%%%%%%%%%%%%%%%%%%%%%%%%%%%%%%%%%%%%%%%%%%%%%%%%%%%%%%%%%%%%%%%%%%
\subsection{Incoming Webhookの設定}
Incoming Webhookとは外部サービスからSlackにメッセージを投稿する機能である.

以下にIncomig Webhookの設定の手順を示す.
\begin{enumerate}
  \item 自身のSlackアカウントにログインし,ワークスペースの設定から,
       「その他の管理項目」,「アプリを管理する」,「カスタムインテグレーション」,「Incoming Webhook」,
       「Slackに追加」の順にアクセスする.
  \item 「チャンネルへの投稿」の項目で,投稿するチャンネルを指定する.
  \item Incoming Webhook URLをSlackBotディレクトリ内のsettings.ymlファイルに以下のように記述する.
        ただし,\verb|<Incoming Webhook URL>|は自身のIncoming Webhook URLである.\\
        \verb|INCOMING_WEBHOOK_URL : <Incomig Webhook URL> |
  \item 「設定を保存」をクリックする.
\end{enumerate}


%%%%%%%%%%%%%%%%%%%%%%%%%%%%%%%%%%%%%%%%%%%%%%%%%%%%%%%%%%%%%%%%%%%%%%%%%%%%%%%%%
\subsection{Google Places APIの設定}
本プログラムはGoogle Places APIを用いているため,Googleが発行するAPIキーがなければ利用することができない.

以下の手順でPlaces APIの設定を行う.
\begin{enumerate}
\item ブラウザで以下のサイトにアクセスし,自身のGoogleアカウントにログインする.\\
      \verb|https://cloud.google.com/maps-platform/|
\item 右上の「開始」ボタンをクリックし,「新しいプロジェクトの作成」を選択し,
     「プロジェクトの選択」から「新しいプロジェクト」に進み,プロジェクト名をつけた後「作成」をクリックする.
\item 「請求先アカウントの作成」をクリックし,クレジットカード情報を入力する.
\item 「APIの概要に移動」をクリックし,「Places API」を選択,「有効にする」をクリックする.
\item 「認証情報」を選択し,APIキーが表示されていることを確認し,SlackBotディレクトリ内のsetting.ymlに以下のように追記する.
      ただし,\verb|<Your_API_key>|は自身のAPIキーである.\\
       \verb|API_KEY : <Your_API_key>|
\end{enumerate}


%%%%%%%%%%%%%%%%%%%%%%%%%%%%%%%%%%%%%%%%%%%%%%%%%%%%%%%%%%%%%%%%%%%%%%%%%%
\subsection{Herokuの設定}
Herokuの設定について説明を行っていく.

\begin{enumerate}
  \item 以下のサイトにアクセスし,「Sign up」から新しいアカウントを作成する.\\
        \verb|https://www.heroku.com|
  \item 登録したアカウントにログインし,Rubyを選択し,「I'm ready to start」をクリックする.
  \item 自身のOSを選択し,Heroku CLIをダウンロードする.
  \item 以下のコマンドを実行し,メールアドレスとパスワードを入力してHerokuにログインする.\\
        \verb|$ heroku lonin|
  \item 以下のコマンドを実行して,Heroku上にアプリケーションを作成する.\\
        \verb|$ heroku create|\\
        ここで表示されたHerokuアプリケーションのURLがSlackBotサーバのURLとなる.
        これは後の設定で用いるため,控えておく.
  \item 以下のコマンドを順次実行し,heroku上にSlackBotプログラムをデプロイする.\\
        \verb|$ git add --all|\\
        \verb|$ git commit -m "First deploy."|\\
        \verb|$ git push origin master|
\end{enumerate}


%%%%%%%%%%%%%%%%%%%%%%%%%%%%%%%%%%%%%%%%%%%%%%%%%%%%%%%%%%%%%%%%%%%%%%%%%%%
\subsection{Outgoing Webhookの設定}
Outgoing Webhookとは,Slackに特定の文字列を含む投稿がされると,外部サービスにメッセージを送信する機能である.

以下にOutgoing Webhookの設定の手順を示す. 
\begin{enumerate}
  \item 自身のSlackアカウントにログインし,ワークスペースの設定から,
  「その他の管理項目」,「アプリを管理する」,「カスタムインテグレーション」,「Outgoing Webhook」,
  「Slackに追加」の順にアクセスする.
  \item 「チャンネルへの投稿」の項目で,投稿するチャンネルを指定する. 
  \item 引き金となる単語を設定する.
  \item URLにHerokuの設定で控えたSlackBotサーバのURLを入力する.
  \item 「設定を保存」をクリックする.
\end{enumerate}


%%%%%%%%%%%%%%%%%%%%%%%%%%%%%%%%%%%%%%%%%%%%%%%%%%%%
%%%%%%%%%%%%%%%%%%%%%%%%%%%%%%%%%%%%%%%%%%%%%%%%%%%%
\section{使用方法}
Slack上で指定したチャンネルで,\ref{sec:function}章で示した発言をすることで,自動的にSlackBotの返信が行われる.

%%%%%%%%%%%%%%%%%%%%%%%%%%%%%%%%%%%%%%%%%%%%%%%%%%%
%%%%%%%%%%%%%%%%%%%%%%%%%%%%%%%%%%%%%%%%%%%%%%%%%%%
\section{エラー処理}
本プログラムで実装しているエラー処理に関して以下に示す.

\begin{enumerate}
  \item 郵便番号を検索する機能に関して,郵便番号以外の値が入力された場合への対処.
        この場合,以下のようなメッセージを返す.\\
        Invalid Zipcode
  \item コンビニを検索する機能に関して,表示できるコンビニが存在しない場合への対処.
        この場合,以下のようなメッセージを返す.\\
        No hit!
  \item 無効なリクエストを送信した場合や,APIとの接続に失敗した場合への対処.
        この場合,それぞれ以下のようにメッセージを返す.\\
        郵便番号を検索する機能:Failed to exchange zipcode to address.\\
        コンビニを検索する機能:Failed to search convenience\_store.
\end{enumerate}

%%%%%%%%%%%%%%%%%%%%%%%%%%%%%%%%%%%%%%%%%%%%%%%%%%%
%%%%%%%%%%%%%%%%%%%%%%%%%%%%%%%%%%%%%%%%%%%%%%%%%%%
\section{保証しない動作}
本プログラムは以下に示す状況に関して動作の保証をしていない.

\begin{enumerate}
  \item コンビニを検索する機能で,件数に負の数や,小数が入力された場合.
\end{enumerate}

%%%%%%%%%%%%%%%%%%%%%%%%%%%%%%%%%%%%%%%%%%%%%%%%%%%
%%%%%%%%%%%%%%%%%%%%%%%%%%%%%%%%%%%%%%%%%%%%%%%%%%%
\section{おわりに}
\label{sec:conclusion}
本資料では新人研修課題のSlackBotプログラムの仕様を説明した.

%\bibliographystyle{ipsjunsrt}
%\bibliography{mybibdata}

\end{document}
